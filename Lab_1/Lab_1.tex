\documentclass[a4paper,14pt]{article}

% Preamble
\usepackage[utf8]{inputenc}
\usepackage[T1]{fontenc}
\usepackage{lmodern} 

\begin{document}
	
	\begin{titlepage}
		\centering
		{\scshape\LARGE Matemática estructural y Lógica: Laboratorio 1 \par}
		\vspace{0.5cm}
		{\scshape\Large Andrés Darío Sánchez -- 201218388 \par}
		\vspace{2cm}
		% Avoid pagebreak after end of titlepage
		\let\endtitlepage\relax
	\end{titlepage}
	
	\section{Demostrar: $(p \lor q) \land (\neg p \land (\neg p \land q)) \equiv \neg p \land q$}
	\textbf{Teo:} $(p \lor q) \land (\neg p \land (\neg p \land q)) \equiv \neg p \land q$ \newline
	\textbf{Dem:} \newline 
	\[ (p \lor q) \land (\neg p \land (\neg p \land q)) \]
	\[<DeMorgan>\]
	
	\section{}
	Considere el operador ternario A definido por el siguiente axioma:
	\[ Def A: A(x,y,z) \equiv (x \rightarrow y \lor z) \]
	
	\noindent\textbf{Teo: $A(p \lor q, q, r) \equiv A(p, q, r)$} \newline
	\textbf{Dem: } 
	\[ A(p \lor q, q, r) \]
	\[<Def A> \]
	
	\vspace{1cm}\noindent\textbf{Teo: $A(p, q, r) \equiv A(p, q, false) \lor A(p, r, false)$} \newline
	\textbf{Dem:}
	\[ <Def A> \]
	
	
\end{document}